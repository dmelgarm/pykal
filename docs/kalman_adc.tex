%%%%%%%%%%%%%%%%%%%%%%%%%%%%%%%%%%%%%%%%%
% Short Sectioned Assignment
% LaTeX Template
% Version 1.0 (5/5/12)
%
% This template has been downloaded from:
% http://www.LaTeXTemplates.com
%
% Original author:
% Frits Wenneker (http://www.howtotex.com)
%
% License:
% CC BY-NC-SA 3.0 (http://creativecommons.org/licenses/by-nc-sa/3.0/)
%
%%%%%%%%%%%%%%%%%%%%%%%%%%%%%%%%%%%%%%%%%

%----------------------------------------------------------------------------------------
%	PACKAGES AND OTHER DOCUMENT CONFIGURATIONS
%----------------------------------------------------------------------------------------

\documentclass[paper=a4, fontsize=11pt]{scrartcl} % A4 paper and 11pt font size

\usepackage[T1]{fontenc} % Use 8-bit encoding that has 256 glyphs
\usepackage{fourier} % Use the Adobe Utopia font for the document - comment this line to return to the LaTeX default
\usepackage[english]{babel} % English language/hyphenation
\usepackage{amsmath,amsfonts,amsthm} % Math packages

\usepackage{lipsum} % Used for inserting dummy 'Lorem ipsum' text into the template

\usepackage{sectsty} % Allows customizing section commands
\allsectionsfont{\centering \normalfont\scshape} % Make all sections centered, the default font and small caps

\usepackage{fancyhdr} % Custom headers and footers
\pagestyle{fancyplain} % Makes all pages in the document conform to the custom headers and footers
\fancyhead{} % No page header - if you want one, create it in the same way as the footers below
\fancyfoot[L]{} % Empty left footer
\fancyfoot[C]{} % Empty center footer
\fancyfoot[R]{\thepage} % Page numbering for right footer
\renewcommand{\headrulewidth}{0pt} % Remove header underlines
\renewcommand{\footrulewidth}{0pt} % Remove footer underlines
\setlength{\headheight}{13.6pt} % Customize the height of the header

\numberwithin{equation}{section} % Number equations within sections (i.e. 1.1, 1.2, 2.1, 2.2 instead of 1, 2, 3, 4)
\numberwithin{figure}{section} % Number figures within sections (i.e. 1.1, 1.2, 2.1, 2.2 instead of 1, 2, 3, 4)
\numberwithin{table}{section} % Number tables within sections (i.e. 1.1, 1.2, 2.1, 2.2 instead of 1, 2, 3, 4)

\setlength\parindent{0pt} % Removes all indentation from paragraphs - comment this line for an assignment with lots of text

%----------------------------------------------------------------------------------------
%	TITLE SECTION
%----------------------------------------------------------------------------------------

\newcommand{\horrule}[1]{\rule{\linewidth}{#1}} % Create horizontal rule command with 1 argument of height

\title{	
\normalfont \normalsize 
\textsc{Scripps Institution of Oceanography} \\ [25pt] % Your university, school and/or department name(s)
\horrule{0.5pt} \\[0.4cm] % Thin top horizontal rule
\huge Extending the Kalman Filter to Accelerometers with a DC Offset \\ % The assignment title
\horrule{2pt} \\[0.5cm] % Thick bottom horizontal rule
}

\author{Diego Melgar} % Your name

\date{\normalsize\today} % Today's date or a custom date

\begin{document}

\maketitle % Print the title

%----------------------------------------------------------------------------------------
%	PROBLEM 1
%----------------------------------------------------------------------------------------

\section{Filter Formulation}

The problem is to modify the Kalman filtered from \textit{Bock et al.} (2011) to a filter that can ingest accelerometer data with a constant DC offset. That is the measured or observed acceleration $a^{obs}$ is related to the true acceleration, $a^{true}$ by
$$
a_k^{true}=a_k^{obs}-\Omega_k+\epsilon_a
$$
where $\Omega_k$ is the DC offset at epoch $k$ and $\epsilon_a$ is the accelerometer noise. If we define our system states as the displacement $d$, the velocity $v$ and the DC offset $\Omega$ then following \textit{Lewis et al.} (2008) section 2.4 we can write the continuous difference equation for this system:
$$
\frac{d}{dt}
\left[\begin{matrix}
  d(t) \\
  v(t) \\
  \Omega(t)
\end{matrix}\right]
=\frac{d}{dt}\mathbf{x}(t)
=\mathbf{A}(t)\mathbf{x}(t)+\mathbf{B}(t)u(t)+\mathbf{\epsilon}(t)
$$
where
$$
\mathbf{A}=
\left[\begin{matrix}
0 & 1 & 0 \\
0 & 0 & -1 \\
0 & 0 & 0
\end{matrix}\right]
\;;\;
\mathbf{B}=
\left[\begin{matrix}
0 \\
1 \\
0
\end{matrix}\right]\;;\;u=a^{obs}\;;\;\epsilon=
\left[\begin{matrix}
0 \\
\epsilon_a\\
\epsilon_\Omega
\end{matrix}\right]\;,
$$
where $\epsilon_a$ is the accelerometer noise and $\epsilon_\Omega$ is the DC offset noise. A small value of $\epsilon_\Omega$ will allow the DC offset to vary slowly through time. This means that the noise vector $\epsilon$ is Gaussian, such that $\epsilon\sim(0,\mathbf{Q})$ where the covariance $\mathbf{Q}$ depends on the accelerometer and DC offset noise variances $\sigma_a$ and $\sigma_\Omega$ like
$$
Q=\left[\begin{matrix}
 0 & 0 & 0 \\
 0 & \sigma_a & 0 \\
 0 & 0 & \sigma_\Omega
\end{matrix}\right]
$$
You can expand the last two equations to verify the system:
$$
\frac{d}{dt}
\left[\begin{matrix}
  d(t) \\
  v(t) \\
  \Omega(t)
\end{matrix}\right]=
\left[\begin{matrix}
  \dot{d} \\
  \dot{v} \\
  \dot{\Omega}
\end{matrix}\right]=
\left[\begin{matrix}
  v\\
  -\Omega+a^{obs}+\epsilon_a \\
  \epsilon_\Omega
\end{matrix}\right]\;.
$$
If we are measuring noisy displacements we define the measurement process:
$$
z(t)=d^{obs}(t)=\mathbf{H}(t)\mathbf{x}(t)+\eta_d=
\left[\begin{matrix}
 1 & 0 & 0
\end{matrix}\right]\mathbf{x}(t)+\eta_d
$$
where $\eta_d$ is the displacement observation noise, which when assumed gaussian has a distribution $\mathbf{\eta_d}\sim(0,\mathbf{R})$ with $\mathbf{R}=\sigma_d$ the displacement noise variance. The continuous system can now be discretized. Again following \textit{Lewis et al.}(2008), we write the discrete system:
$$
\mathbf{x}_{k+1}=\mathbf{A}^s\mathbf{x}_k+\mathbf{B}^sa^{obs}_k+\epsilon_k
$$
where the discretized noise vector is Gaussian and distributed like $\epsilon_k\sim(0,\mathbf{Q}^s)$. If we discretize the continuous system at the sampling rate $\tau_a$ of the accelerometer we must redefine the system transition matrices. These are obtained (\textit{Lewis et al.}, 2008) from the MacLaurin series expansion of the integral forms as:
$$
\mathbf{A}^s=\mathbf{I}+\mathbf{A}\tau_a+\frac{\mathbf{A}^2\tau^2}{2}=\left[\begin{matrix}
1 & \tau_a & -\tau_a^2/2 \\
0 & 1 & -\tau_a \\
0 & 0 & 1 \end{matrix}\right]\;;
$$
$$
\mathbf{B}^s=\mathbf{B}\tau_a+\frac{\mathbf{A}\mathbf{B}\tau_a^2}{2}=\left[\begin{matrix}
\tau^2_a/2 \\
\tau_a \\
0 \end{matrix}\right]\;;
$$
$$
\mathbf{Q}^s=\mathbf{Q}\tau_a+\frac{1}{2}(\mathbf{AQ}+\mathbf{QA}^T)\tau_a^2+\frac{1}{3}\mathbf{AQA}^T\tau^3_a=\left[\begin{matrix}
\sigma_a\tau_a^3/3 & \sigma_a\tau_a^2/2 & 0 \\
\sigma_a\tau_a^2/2 & \sigma_a\tau_a+\sigma_\Omega\tau_a^3/3 & -\sigma_\Omega\tau_a^2/2 \\
0 & -\sigma_\Omega\tau_a^2/2 & \sigma_\Omega\tau_a \end{matrix}\right]\;.
$$
The sampled version of the continuous measurement is written as:
$$
z_k=d^{obs}_k=\mathbf{H}^s\mathbf{x}_k+\eta_d\;,
$$
where the noise is is white and Gaussian such that $\eta_d\sim(0,R^s)$. Sampling at the rate of the GPS, $\tau_d$ the discretized matrices are simply:
$$
\mathbf{H}^s=\mathbf{H}=\left[\begin{matrix}
 1 & 0 & 0
\end{matrix}\right]
$$
$$
R^s=\sigma_d/\tau_d\;.
$$
With these definitions the traditional system update and measurement update stages of the Kalman filter can be carried out.


\end{document}